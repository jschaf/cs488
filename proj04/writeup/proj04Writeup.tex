\documentclass[12pt,letterpaper,titlepage]{article}

\pdfpagewidth 8.5in
\pdfpageheight 11in 

% 1in margins
\usepackage[vmargin={1.6in, 0.3in}, hmargin={1in, 1in}]{geometry}
\usepackage{eulervm}
\usepackage{setspace}
\usepackage{fancyhdr}


% Minion support
% \usepackage[opticals,minionint,textosf,mathlf]{MinionPro}

% XeTeX
\usepackage[no-math]{fontspec}
\usepackage{xunicode,xltxtra}


\usepackage{underscore}
% To get proper looking page headers
\setlength{\voffset}{-0.5in}
\setlength{\headsep}{12pt}
\setlength{\headheight}{15.2pt}
\pagestyle{fancy}
\fancyhead{}
\fancyfoot{}
\renewcommand\headrulewidth{0pt}
\fancyhead[R]{Schafer \thepage}

% Line order is important for these
\defaultfontfeatures{Scale=MatchLowercase,Mapping=tex-text}
\setromanfont[Ligatures={Common}]{Adobe Garamond Pro}
\setsansfont{Akzidenz Grotesk Roman} 
\setmonofont[Scale=0.8]{Consolas}

% Use diffferent styles for section headings
\usepackage{sectsty} 
\sectionfont{\sffamily\mdseries\large} 
\subsectionfont{\rmfamily\mdseries\scshape\normalsize} 

\setcounter{secnumdepth}{0}

\renewcommand{\th}[0]{\textsuperscript{th}}
\newcommand{\st}[0]{\textsuperscript{st}}
\newcommand{\nd}[0]{\textsuperscript{nd}}
\newcommand{\rd}[1]{\textsuperscript{rd}}
\newcommand{\code}[1]{\texttt{#1}}
\newcommand{\mieds}[0]{\textsc{MIEDS}}
\begin{document}

\begin{titlepage}
  \begin{center}
    \singlespacing
    UNITED STATES MILITARY ACADEMY
    \vfill
    
    PROJECT 4

    \vfill
    
    CS488: LANG-BASED SIMULATION MODELING\\[12pt]
    SECTION C1\\[12pt]
    COLONEL GENE RESSLER

    \vfill
    
    BY\\[12pt]
    CDT JOE SCHAFER '10, CO F3

    \vfill
    
    WEST POINT, NEW YORK \\[12pt]
    12 MAY 2010
  \end{center}

  \vfill

  \begin{flushleft}
    \rule{24pt}{0.5pt}
    MY DOCUMENTATION IDENTIFIES ALL SOURCES USED AND ASSISTANCE
    RECEIVED IN COMPLETING THIS ASSIGNMENT
    \\[12pt]
    
    \rule{24pt}{0.5pt}
    NO SOURCES WERE USED OR ASSISTANCE RECEIVED IN COMPLETING THIS
    ASSIGNMENT
    \\[18pt]

    SIGNATURE: \rule{30em}{0.5pt}\\[12pt]
  \end{flushleft}
\end{titlepage}

\doublespacing

\section{Code Changes}

To record the number of successful and total trips, and the number of
detonated hazards and total hazards I created a new record with type
\code{Completion_Data_Type}.  The record simply contains four
\code{Natural} fields corresponding to the information we wish to
track.  In the \code{Simulation_State_Type}, I added a new field
entitled \code{Completion_Data} to allow easy access for updates.  The
logical place to update such information is in the polymorphic
procedure \code{Handle} which operates on \code{Friend_Movement_Types}
because \code{Handle} checks for hazard collisions and completed
trips.  So after the check for \code{Hazard_ID > 0} (a hazard
collision occured) and the check to \code{Friend_Is_Hurt}, we call the
convience procedure, \code{Log_Unsuccessful_Trip} with the current
state to increment the number of total trips and the number of
detonated hazard.  When a trip is complete, we use the convenience
procedure \code{Log_Successful_Trip} to increment the number of
successful trips and total trips.  Finally, we need to track the total
number of hazards, which occurs in \code{Handle} for
\code{Hazard_Emplacement_Type}.  We use the convience procedure
\code{Log_Hazard_Emplacment} to increment the number of total hazards
and call this in the opening lines of \code{Handle}

\section{Impact of Point Estimators on Run Length}
\section{Change in the confidence half-interval over the change in number of runs}

\section{Why are the number of hazards placed unchanging over independent replications?}

\section{How many runs are truly different between 2 runs and 30 runs in the current simulator?}

\section{Output Data}


\end{document}
