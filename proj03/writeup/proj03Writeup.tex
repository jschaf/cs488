\documentclass[12pt,letterpaper,titlepage]{article}

\pdfpagewidth 8.5in
\pdfpageheight 11in 

% 1in margins
\usepackage[vmargin={1.6in, 0.3in}, hmargin={1in, 1in}]{geometry}
\usepackage{fontspec,xunicode,xltxtra}
\usepackage{setspace}
\usepackage{fancyhdr}

\usepackage{underscore}
% To get proper looking page headers
\setlength{\voffset}{-0.5in}
\setlength{\headsep}{12pt}
\setlength{\headheight}{15.2pt}
\pagestyle{fancy}
\fancyhead{}
\fancyfoot{}
\renewcommand\headrulewidth{0pt}
\fancyhead[R]{Schafer \thepage}

% Line order is important for these
\defaultfontfeatures{Mapping=tex-text}
\setromanfont[Ligatures={Common}]{Adobe Garamond Pro}
\setsansfont{Akzidenz Grotesk Roman} 
\setmonofont[Scale=0.8]{Consolas}

% Use diffferent styles for section headings
\usepackage{sectsty} 
\sectionfont{\sffamily\mdseries\large} 
\subsectionfont{\rmfamily\mdseries\scshape\normalsize} 

\setcounter{secnumdepth}{0}

\renewcommand{\th}[0]{\textsuperscript{th}}
\newcommand{\st}[0]{\textsuperscript{st}}
\newcommand{\nd}[0]{\textsuperscript{nd}}
\newcommand{\rd}[1]{\textsuperscript{rd}}
\newcommand{\code}[1]{\texttt{#1}}
\newcommand{\mieds}[0]{\textsc{MIEDS}}
\begin{document}

\begin{titlepage}
  \begin{center}
    \singlespacing
    UNITED STATES MILITARY ACADEMY
    \vfill
    
    PROJECT 3

    \vfill
    
    CS488: LANG-BASED SIMULATION MODELING\\[12pt]
    SECTION C1\\[12pt]
    COLONEL GENE RESSLER

    \vfill
    
    BY\\[12pt]
    CDT JOE SCHAFER '10, CO F3

    \vfill
    
    WEST POINT, NEW YORK \\[12pt]
    3 MAY 2010
  \end{center}

  \vfill

  \begin{flushleft}
    \rule{24pt}{0.5pt}
    MY DOCUMENTATION IDENTIFIES ALL SOURCES USED AND ASSISTANCE
    RECEIVED IN COMPLETING THIS ASSIGNMENT
    \\[12pt]
    
    \rule{24pt}{0.5pt}
    NO SOURCES WERE USED OR ASSISTANCE RECEIVED IN COMPLETING THIS
    ASSIGNMENT
    \\[18pt]

    SIGNATURE: \rule{30em}{0.5pt}\\[12pt]
  \end{flushleft}
\end{titlepage}

\doublespacing

\section{Simulator Description}

The \mieds{} simulator works by initializing the animation, scheduling
the first events, and then looping through all events, processing each
and updating the animation, until either no events remain or time is
exceeded.  The key data structure is \code{Simulation_State_Type}, a
record containing the current time, event queue, AST of the \mieds
source code, emplaced hazards, and currently enroute friends.

\subsection{Scheduling Initial Events}

\code{Schedule_Initial_Events} works by processing instances in the
model section of the \mieds{} sourcecode.  The instance is either a
trip or threat.  First, the start time is calculated with
\code{Find_Trip_Schedule_Start} or \code{Find_Threat_Schedule_Start},
both of which simply lookup start times contained in the associated
records.  Then, the event is added to the simulation state using
\code{Schedule_Trip_Start} or \code{Schedule_Hazard}.  This process is
repeated for all instances in the \mieds{} source code.

\subsection{Looping and Processing Events}

The simulation runs until the simulation state time exceeds the stop
time, which is defined as the last real number
(i.e. \code{Real'Last}), or until the simulation state event queue is
empty.  During each iteration, first, the next event is popped from
the queue by \code{Get_Next_Event}.  Second, the state simulation time
is updated to match the current event's time.  Third, the current
event is processed by calling the polymorphic procedure \code{Handle},
which is defined for trip start, friend movement, hazard emplacement
and hazard removal events.  Fourth, the memory for the event is
deallocated.  Finally, the animation is updated with current
simulation state.

\subsection{Event Handlers}
The trip start \code{Handle} procedure appends the trip start event to
the simulation state \code{Friend_Enroute} list, advances the ID for
the next friend enroute event, schedules the friend movement and
schedules the next trip start by adding the trip start interval to the
event time.

The friend movement\code{Handle} procedure calculates the native speed
of the friend, advances the friend along its route, and checks for
trip completion.  If complete, then the event is deleted, else the
friend enroute information is updated and the next movement is
scheduled and placed in the simulation state event queue.
\code{Advance} is most complex part of friend movement, requiring ten
parameters.  In order of operation \code{Advance}, computes the inital
segement, checks for stoppers, checks for route completion, updates
the friend position and checks for stoppage during a partial
traversal.


The hazard emplacement \code{Handle} determines the duration of the
hazard, then initializes and randomly places the hazard along a
segment.  If the duration of the is greater than zero, an empty hazard
emplacement event is appended to the hazards list in the simulation
state and replaced by the actual hazard after the hazard removal is
scheduled.  If the duration was not greater than zero, then the event
is appended and no removal is scheduled.  Finally, the hazard interval
is computed if necessary.

The hazard removal \code{Handle} simply deletes the hazard from the
hazards list in the simulation state

\subsection{Event Deallocation}

An instance of \code{Ada.Unchecked_Deallocation} named \code{Free} is
used to delete already processed events to prevent memory leaks.


\section{Changed Data Structures and Code}

Explain

\subsection{Changes to Data Structures}
1. changes to data structures


\subsection{Changes to Code}
2. changes to code that must be made in order to implement the desired
simulator function.

Again be as specific as possible by discussing types, procedures, and
algorithms in detail.


\end{document}
