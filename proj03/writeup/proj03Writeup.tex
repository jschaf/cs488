\documentclass[12pt,letterpaper,titlepage]{article}

\pdfpagewidth 8.5in
\pdfpageheight 11in 

% 1in margins
\usepackage[vmargin={1.6in, 0.3in}, hmargin={1in, 1in}]{geometry}
\usepackage{eulervm}
\usepackage{setspace}
\usepackage{fancyhdr}


% Minion support
% \usepackage[opticals,minionint,textosf,mathlf]{MinionPro}

% XeTeX
\usepackage[no-math]{fontspec}
\usepackage{xunicode,xltxtra}


\usepackage{underscore}
% To get proper looking page headers
\setlength{\voffset}{-0.5in}
\setlength{\headsep}{12pt}
\setlength{\headheight}{15.2pt}
\pagestyle{fancy}
\fancyhead{}
\fancyfoot{}
\renewcommand\headrulewidth{0pt}
\fancyhead[R]{Schafer \thepage}

% Line order is important for these
\defaultfontfeatures{Scale=MatchLowercase,Mapping=tex-text}
\setromanfont[Ligatures={Common}]{Adobe Garamond Pro}
\setsansfont{Akzidenz Grotesk Roman} 
\setmonofont[Scale=0.8]{Consolas}

% Use diffferent styles for section headings
\usepackage{sectsty} 
\sectionfont{\sffamily\mdseries\large} 
\subsectionfont{\rmfamily\mdseries\scshape\normalsize} 

\setcounter{secnumdepth}{0}

\renewcommand{\th}[0]{\textsuperscript{th}}
\newcommand{\st}[0]{\textsuperscript{st}}
\newcommand{\nd}[0]{\textsuperscript{nd}}
\newcommand{\rd}[1]{\textsuperscript{rd}}
\newcommand{\code}[1]{\texttt{#1}}
\newcommand{\mieds}[0]{\textsc{MIEDS}}
\begin{document}

\begin{titlepage}
  \begin{center}
    \singlespacing
    UNITED STATES MILITARY ACADEMY
    \vfill
    
    PROJECT 4

    \vfill
    
    CS488: LANG-BASED SIMULATION MODELING\\[12pt]
    SECTION C1\\[12pt]
    COLONEL GENE RESSLER

    \vfill
    
    BY\\[12pt]
    CDT JOE SCHAFER '10, CO F3

    \vfill
    
    WEST POINT, NEW YORK \\[12pt]
    12 MAY 2010
  \end{center}

  \vfill

  \begin{flushleft}
    \rule{24pt}{0.5pt}
    MY DOCUMENTATION IDENTIFIES ALL SOURCES USED AND ASSISTANCE
    RECEIVED IN COMPLETING THIS ASSIGNMENT
    \\[12pt]
    
    \rule{24pt}{0.5pt}
    NO SOURCES WERE USED OR ASSISTANCE RECEIVED IN COMPLETING THIS
    ASSIGNMENT
    \\[18pt]

    SIGNATURE: \rule{30em}{0.5pt}\\[12pt]
  \end{flushleft}
\end{titlepage}

\doublespacing

\section{Simulator Description}

The \mieds{} simulator works by initializing the animation, scheduling
the first events, and then looping through all events, processing each
and updating the animation, until either no events remain or time is
exceeded.  The key data structure is \code{Simulation_State_Type}, a
record containing the current time, event queue, AST of the \mieds{}
source code, emplaced hazards, and currently enroute friends.

\subsection{Scheduling Initial Events}

\code{Schedule_Initial_Events} works by processing instances in the
model section of the \mieds{} source code.  The instance is either a
trip or threat.  First, the start time is calculated with
\code{Find_Trip_Schedule_Start} or \code{Find_Threat_Schedule_Start},
both of which simply lookup start times contained in the associated
records.  Then, the event is added to the simulation state using
\code{Schedule_Trip_Start} or \code{Schedule_Hazard} which inserts the
event into \code{State.Event_Queue}.  This process is repeated for all
instances in the \mieds{} source code.

\subsection{Looping and Processing Events}

The simulation runs until the simulation state time exceeds the stop
time, defined by default as the last real number (\code{Real'Last}),
or until \code{State.Event_Queue} is empty.  Five stages comprise each
iteration.  First, the next event is popped from the queue by
\code{Get_Next_Event}.  Second, \code{State.Time} is set to the
current \code{Event.Time}.  Third, the current event is processed by
calling the polymorphic procedure \code{Handle}, which is defined for
\code{Trip_Start_Type}, \code{Friend_Movement_Type},
\code{Hazard_Emplacement_Type} and \code{Hazard_Removal_Type} events.
Fourth, the memory for the event is deallocated.  Finally, the
animation is updated with current simulation state.

\subsection{Event Handlers}
The trip start \code{Handle} procedure first appends the trip start
event to \code{State.Friend_Enroute}, a doubly linked list of
\code{Friend_Enroute_Type}.  Second, \code{Handle} increments
\code{Next_Friend_Enroute_ID}.  Third, the initial trip movement is
scheduled with \code{Schedule_Friend_Movement}, which allocates a new
\code{Friend_Movement_Type} with the correct time and inserts it into
\code{State.Event_Queue}.  Finally, if the trip is cyclic (i.e. has an
interval greater than 0.0, as calculated by
\code{Find_Trip_Schedule_Interval}) then the next trip is scheduled
for time \code{Event.Time + Interval}.

The friend movement \code{Handle} procedure first calculates the
\code{Native_Speed} of the friend using \code{Find_Friend_Speed},
which lookups the expression from the \mieds{} AST.  Second,
\code{Handle} advances the friend along its route using
\code{Advance}, which ultimately determines where the friend moves.
In order of operation, \code{Advance} computes the inital segement,
checks for stoppers, checks for route completion, updates the friend
position and checks for stoppage during a partial traversal.  If the
friend trip \code{Is_Complete} then it is deleted from
\code{State.Friends_Enroute}, else
\code{Friend_Enroute.Last_Update_Time} is set to the current time, the
friend enroute data is replaced with the new data and the next
movement is scheduled.

The hazard emplacement \code{Handle} first determines the duration of
the hazard with \code{Find_Threat_Duration}.  Next, \code{Handle}
initializes and randomly sets the hazard location with
\code{Get_Random_Point}.  If the hazard is still viable
(\code{Duration > 0.0}), then an empty hazard is created and appended
to \code{State.Hazards} as a placeholder,
\code{Schedule_Hazard_Removal} schedules the removal event for the
hazard, and the placeholder is replaced with the actual hazard.  If
the hazard is not viable and scheduled to be removed, then the hazard
is simply appended to \code{State.Hazards}.  Finally, if the the
hazard is viable and cyclic (i.e. \code{Interval > 0.0}) then the next
cycle is scheduled with \code{Schedule_Hazard} for time
\code{Event.Time + Interval}.

The hazard removal \code{Handle} simply deletes the hazard from
\code{State.Hazards}, a doubly-linked list of emplaced hazards.

\clearpage

\section{Changes to Data Structures}

No changes were made to the original Data Structures, which is
slightly worrisome but a quote by Alan Perlis provides some small
measure of comfort: ``It is better to have 100 functions operate on
one data structure than 10 functions on 10 data structures.''


\section{Changes to Code}
First, we must determine if a friend is within ten meters of a hazard,
a problem commonly known as collision detection.  \code{Advance}
handles collision detection by returning a \code{Stopper_ID} if any
element in \code{Stoppers} is less than \code{Stopper_Radius} away.
We must convert \code{State.Hazards} into a stopper list.  We
accomplish this with the aptly named
\code{Convert_Hazards_To_Stoppers} which iterates through the hazards
and fills an array of the same length with \code{Hazard.ID} and
\code{Hazard.Location} to create a \code{Stopper_Type}.  After
\code{Advance} returns an ID, we convert it back to a
\code{Emplaced_Hazard_Cursor_Type} using a function named
\code{Find_Hazard_From_ID}.  The function works by creating a dummy
\code{Emplaced_Hazard_Type} with the a matching ID and then uses
\code{Find} to get the correct cursor from \code{State.Hazards}.  We
could use a map from IDs to cursors, but that would increase the
complexity for little benefit.  The number of hazards is usually very
small.  With the hazard cursor, we can determine if the friend will be
hurt with \code{Friend_Is_Hurt}.

\subsection{Friend_Is_Hurt Definition}
\code{Friend_Is_Hurt} takes three parameters, \code{Friend_Enroute} as
defined in \code{Handle}, \code{Hazard_Cursor} as provided by
\code{Find_Hazard_From_ID}, and the out parameter \code{Is_Hurt}. To
determine if a friend is hurt by a hazard we need a random number
generator, the friend vulnerability and the hazard effectiveness.
\code{Ada.Numerics.Float_Random} is already in scope and provides
random floats between 0.0 and 1.0.  We simply instaniate a
\code{Generator} and call \code{Random} with the generator to obtain a
random number. \code{Find_Friend_Vulnerability} is already defined.
\code{Find_Threat_Effectiveness} must be defined in the
\code{ast-tree} package as an accessor for the \code{Effectiveness}
field in \code{Threat_Type} by using the convenience function
\code{Find_Positive_Value}.  A friend is hurt if:

\[random(generator) \leq vulnerability \times effectiveness\]

\subsection{Cleaning Up}
After determining a friend is hurt, we must remove the hazard removal
event from \code{State.Event_Queue}, the friend from
\code{State.Friends_Enroute} and the hazard from \code{State.Hazards}.
We remove the hazard removal event by using the
\code{Removal_Event_Cursor} field in the Hazard to delete the
corresponding event.  We remove the friend by calling \code{FEL.Delete}
with \code{State.Friends_Enroute} and \code{Friend_Enroute_Cursor}
which are previously defined.  We remove the hazard in a similar
manner, calling \code{EHL.Delete} with \code{State.Hazards} and the
current hazard cursor.

\end{document}
